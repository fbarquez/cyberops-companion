% BSI IT-Grundschutz Compliance Report Template
% Following DIN 5008 and BSI Standard 200-2 guidelines
% Author: CyberOps Companion Enterprise
\documentclass[a4paper,11pt,ngerman]{article}

% Packages
\usepackage[utf8]{inputenc}
\usepackage[T1]{fontenc}
\usepackage[ngerman]{babel}
\usepackage{geometry}
\usepackage{fancyhdr}
\usepackage{graphicx}
\usepackage{xcolor}
\usepackage{array}
\usepackage{tabularx}
\usepackage{booktabs}
\usepackage{longtable}
\usepackage{multirow}
\usepackage{colortbl}
\usepackage{tikz}
\usepackage{pgfplots}
\usepackage{enumitem}
\usepackage{hyperref}
\usepackage{lastpage}
\usepackage{tocloft}
\usepackage{titlesec}
\usepackage{helvet}

% DIN 5008 Page Setup
\geometry{
    a4paper,
    left=25mm,
    right=20mm,
    top=20mm,
    bottom=20mm,
    headheight=15mm,
    headsep=10mm,
    footskip=15mm
}

% Font settings (Helvetica/Arial as per German business standards)
\renewcommand{\familydefault}{\sfdefault}

% Colors following BSI corporate design
\definecolor{bsiblue}{RGB}{0,51,102}
\definecolor{bsigray}{RGB}{102,102,102}
\definecolor{bsigreen}{RGB}{0,128,0}
\definecolor{bsiyellow}{RGB}{255,193,7}
\definecolor{bsired}{RGB}{220,53,69}
\definecolor{bsilightgray}{RGB}{245,245,245}
\definecolor{tableheader}{RGB}{0,51,102}
\definecolor{tablerow1}{RGB}{255,255,255}
\definecolor{tablerow2}{RGB}{240,244,248}

% Header and Footer
\pagestyle{fancy}
\fancyhf{}
\fancyhead[L]{\small\textcolor{bsigray}{BSI IT-Grundschutz Compliance Report}}
\fancyhead[R]{\small\textcolor{bsigray}{<<CLASSIFICATION>>}}
\fancyfoot[L]{\small\textcolor{bsigray}{<<ORGANIZATION>> | Version <<VERSION>>}}
\fancyfoot[C]{\small\textcolor{bsigray}{Seite \thepage\ von \pageref{LastPage}}}
\fancyfoot[R]{\small\textcolor{bsigray}{<<DATE>>}}
\renewcommand{\headrulewidth}{0.4pt}
\renewcommand{\footrulewidth}{0.4pt}

% Section formatting
\titleformat{\section}
    {\Large\bfseries\color{bsiblue}}
    {\thesection}{1em}{}
\titleformat{\subsection}
    {\large\bfseries\color{bsiblue}}
    {\thesubsection}{1em}{}
\titleformat{\subsubsection}
    {\normalsize\bfseries\color{bsiblue}}
    {\thesubsubsection}{1em}{}

% Custom commands
\newcommand{\compliant}{\textcolor{bsigreen}{\textbf{Konform}}}
\newcommand{\partial}{\textcolor{bsiyellow}{\textbf{Teilweise}}}
\newcommand{\gap}{\textcolor{bsired}{\textbf{Luecke}}}
\newcommand{\notapplicable}{\textcolor{bsigray}{\textbf{N/A}}}

% Compliance score bar
\newcommand{\scorebar}[1]};
    \end{tikzpicture}%
}

\begin{document}

% Title Page
\begin{titlepage}
    \centering
    \vspace*{2cm}

    % Logo placeholder
    \begin{tikzpicture}
        \fill[bsiblue] (0,0) rectangle (3,1);
        \node[white,font=\bfseries\Large] at (1.5,0.5) {BSI};
    \end{tikzpicture}

    \vspace{1cm}

    {\Huge\bfseries\color{bsiblue} BSI IT-Grundschutz\\[0.3cm]
    Compliance Report}

    \vspace{1cm}

    {\Large nach BSI-Standard 200-2}

    \vspace{2cm}

    \begin{tabular}{ll}
        \textbf{Organisation:} & <<ORGANIZATION>> \\[0.3cm]
        \textbf{Berichtszeitraum:} & <<PERIOD>> \\[0.3cm]
        \textbf{Schutzbedarf:} & <<SCHUTZBEDARF>> \\[0.3cm]
        \textbf{Erstellungsdatum:} & <<DATE>> \\[0.3cm]
        \textbf{Version:} & <<VERSION>> \\[0.3cm]
        \textbf{Klassifizierung:} & <<CLASSIFICATION>> \\
    \end{tabular}

    \vfill

    \begin{tikzpicture}
        \fill[bsiblue] (0,0) rectangle (16,0.1);
    \end{tikzpicture}

    \vspace{0.5cm}

    {\small Erstellt mit CyberOps Companion Enterprise\\
    Dieses Dokument ist vertraulich zu behandeln.}

\end{titlepage}

% Table of Contents
\tableofcontents
\newpage

% Document History
\section*{Dokumentenhistorie}
\addcontentsline{toc}{section}{Dokumentenhistorie}

\begin{tabularx}{\textwidth}{|c|c|c|X|}
\hline
\rowcolor{tableheader}
\textcolor{white}{\textbf{Version}} & \textcolor{white}{\textbf{Datum}} & \textcolor{white}{\textbf{Autor}} & \textcolor{white}{\textbf{Aenderungen}} \\
\hline
<<VERSION>> & <<DATE>> & System & Automatisch generiert \\
\hline
\end{tabularx}

\vspace{1cm}

% Distribution List
\section*{Verteiler}
\addcontentsline{toc}{section}{Verteiler}

\begin{tabularx}{\textwidth}{|l|l|X|}
\hline
\rowcolor{tableheader}
\textcolor{white}{\textbf{Name}} & \textcolor{white}{\textbf{Rolle}} & \textcolor{white}{\textbf{Zugriffsrecht}} \\
\hline
<<DISTRIBUTION_LIST>>
\hline
\end{tabularx}

\newpage

% Executive Summary
\section{Management Summary}

\subsection{Ueberblick}

Dieser Bericht dokumentiert den Compliance-Status der Organisation \textbf{<<ORGANIZATION>>}
gegenueber dem BSI IT-Grundschutz Kompendium Edition 2023. Die Bewertung erfolgte
auf Basis des Schutzbedarfs \textbf{<<SCHUTZBEDARF>>}.

\subsection{Gesamtergebnis}

\begin{center}
\begin{tikzpicture}
    % Score circle
    \fill[bsilightgray] (0,0) circle (2.5);
    \fill[bsiblue] (0,0) -- (0,2.5) arc (90:90-<<SCORE>>*3.6:2.5) -- cycle;
    \fill[white] (0,0) circle (1.8);
    \node[font=\Huge\bfseries,color=bsiblue] at (0,0) {<<SCORE>>\%};
    \node[font=\small,color=bsigray] at (0,-0.7) {Compliance Score};
\end{tikzpicture}
\end{center}

\vspace{0.5cm}

\begin{center}
\begin{tabularx}{0.8\textwidth}{|X|c|c|}
\hline
\rowcolor{tableheader}
\textcolor{white}{\textbf{Status}} & \textcolor{white}{\textbf{Anzahl}} & \textcolor{white}{\textbf{Prozent}} \\
\hline
\rowcolor{tablerow1}
\textcolor{bsigreen}{Konform} & <<COMPLIANT_COUNT>> & <<COMPLIANT_PERCENT>>\% \\
\hline
\rowcolor{tablerow2}
\textcolor{bsiyellow}{Teilweise konform} & <<PARTIAL_COUNT>> & <<PARTIAL_PERCENT>>\% \\
\hline
\rowcolor{tablerow1}
\textcolor{bsired}{Luecken} & <<GAP_COUNT>> & <<GAP_PERCENT>>\% \\
\hline
\rowcolor{tablerow2}
\textcolor{bsigray}{Nicht bewertet} & <<NOT_EVALUATED_COUNT>> & <<NOT_EVALUATED_PERCENT>>\% \\
\hline
\rowcolor{tablerow1}
\textcolor{bsigray}{Nicht anwendbar} & <<NOT_APPLICABLE_COUNT>> & <<NOT_APPLICABLE_PERCENT>>\% \\
\hline
\end{tabularx}
\end{center}

\subsection{Bewertung}

<<EXECUTIVE_SUMMARY>>

\newpage

% Scope
\section{Geltungsbereich}

\subsection{Organisation}

\begin{tabularx}{\textwidth}{|l|X|}
\hline
\rowcolor{tableheader}
\multicolumn{2}{|l|}{\textcolor{white}{\textbf{Organisationsdaten}}} \\
\hline
\textbf{Name} & <<ORGANIZATION>> \\
\hline
\textbf{Berichtszeitraum} & <<PERIOD>> \\
\hline
\textbf{Schutzbedarf} & <<SCHUTZBEDARF>> \\
\hline
\textbf{Bewertete Bausteine} & <<TOTAL_BAUSTEINE>> \\
\hline
\textbf{Bewertete Anforderungen} & <<TOTAL_ANFORDERUNGEN>> \\
\hline
\end{tabularx}

\subsection{Angewandte Standards}

\begin{itemize}[leftmargin=*]
    \item BSI IT-Grundschutz Kompendium Edition 2023
    \item BSI-Standard 200-1: Managementsysteme fuer Informationssicherheit
    \item BSI-Standard 200-2: IT-Grundschutz-Methodik
    \item BSI-Standard 200-3: Risikoanalyse auf der Basis von IT-Grundschutz
\end{itemize}

\subsection{Schutzbedarf-Definition}

\begin{tabularx}{\textwidth}{|l|X|}
\hline
\rowcolor{tableheader}
\textcolor{white}{\textbf{Niveau}} & \textcolor{white}{\textbf{Beschreibung}} \\
\hline
\textbf{Basis} & Grundlegende Sicherheitsanforderungen (MUSS-Anforderungen) \\
\hline
\textbf{Standard} & Erweiterte Anforderungen (MUSS + SOLLTE) \\
\hline
\textbf{Hoch} & Maximale Sicherheit (MUSS + SOLLTE + KANN) \\
\hline
\end{tabularx}

\newpage

% Category Results
\section{Ergebnisse nach Kategorie}

<<CATEGORY_RESULTS>>

\newpage

% Detailed Findings
\section{Detaillierte Ergebnisse}

<<DETAILED_FINDINGS>>

\newpage

% Gap Analysis
\section{Lueckenanalyse}

\subsection{Identifizierte Luecken}

<<GAP_ANALYSIS>>

\subsection{Empfehlungen}

<<RECOMMENDATIONS>>

\newpage

% Action Plan
\section{Massnahmenplan}

\begin{longtable}{|p{2cm}|p{5cm}|p{2cm}|p{3cm}|p{2cm}|}
\hline
\rowcolor{tableheader}
\textcolor{white}{\textbf{Prioritaet}} & \textcolor{white}{\textbf{Massnahme}} & \textcolor{white}{\textbf{Status}} & \textcolor{white}{\textbf{Verantwortlich}} & \textcolor{white}{\textbf{Zieldatum}} \\
\hline
\endhead
<<ACTION_ITEMS>>
\hline
\end{longtable}

\newpage

% Appendix
\section{Anhang}

\subsection{Abkuerzungsverzeichnis}

\begin{tabularx}{\textwidth}{|l|X|}
\hline
\rowcolor{tableheader}
\textcolor{white}{\textbf{Abkuerzung}} & \textcolor{white}{\textbf{Bedeutung}} \\
\hline
BSI & Bundesamt fuer Sicherheit in der Informationstechnik \\
\hline
ISMS & Information Security Management System \\
\hline
IT-GS & IT-Grundschutz \\
\hline
MUSS & Zwingende Anforderung (Basis-Absicherung) \\
\hline
SOLLTE & Empfohlene Anforderung (Standard-Absicherung) \\
\hline
KANN & Optionale Anforderung (Erhoehter Schutzbedarf) \\
\hline
\end{tabularx}

\subsection{Referenzen}

\begin{itemize}[leftmargin=*]
    \item BSI IT-Grundschutz Kompendium: \url{https://www.bsi.bund.de/IT-Grundschutz}
    \item BSI Standards 200-x: \url{https://www.bsi.bund.de/grundschutz-standards}
\end{itemize}

\vfill

\begin{center}
\begin{tikzpicture}
    \fill[bsiblue] (0,0) rectangle (16,0.1);
\end{tikzpicture}

\vspace{0.3cm}

{\small\textcolor{bsigray}{
Dieses Dokument wurde automatisch generiert von CyberOps Companion Enterprise.\\
Bei Fragen wenden Sie sich an Ihren Sicherheitsbeauftragten.\\
\textbf{Klassifizierung: <<CLASSIFICATION>>}
}}
\end{center}

\end{document}
